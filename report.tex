\documentclass[11pt]{article}
\usepackage{a4, fullpage}
\usepackage{bibtopic}
%\usepackage[small,compact]{titlesec}
\usepackage{float}
\usepackage{amssymb,amsmath}
\usepackage[T1]{fontenc}
\usepackage{graphicx}
\usepackage{multicol}
\restylefloat{table}
%\usepackage{parskip}
%\usepackage{setspace}

%\setlength{\parskip}{0.3cm}
%\setlength{\parindent}{0cm}
%\setlength{\textheight}{10in}
%\setlength{\textwidth}{6.5in}
%\setlength{\parskip}{2pt}
%\addtolength{\oddsidemargin}{-.3in}
%\addtolength{\evensidemargin}{-.3in}
%\addtolength{\topmargin}{-.6in}
%\addtolength{\textwidth}{.6in}

%Crowdsourcing of tasks has become very popular. It is often based on the principle that by having many different contributors to performing a task, an accurate and appropriate result will be achieved. The classical example is wikipedia where popular pages tend to be more accurate than less popular one. But contributions to a task do not always come for free and frameworks such as Amazon's mechanical turk allow for monetary rewards to be paid in exchange for contributing or performing a task. It is therefore the case that fewer contributions from more trustworthy contributors is more effective than many contributions from less trustworthy sources. The aim of the project is to design and implement a framework that allows the optimum number of contributors to be selected on the basis of their trustworthiness for a desired accuracy of the outcome/result and that evaluates the trustworthiness of contributors on the basis of the accuracy of the results they provide.


\begin{document}



\title{Crowdtrust\\ Group Project }

\author{Giovanni Charles \and Adam Fiksen \and Ryan Jackson \and John Walker}

\date{\today}         % inserts today's date

\maketitle           % generates the title from the data above

%Set the scene ( motivation )
%State the problem you are trying to solve ( Ojective(s) )
%Summarise what you achieved ( contributions )
\section{Introduction}
\subsection{Crowdsourcing and the problems it faces}
Our project concerns \emph{crowdsourcing} 'the book' would describe crowdsourcing as the principle of obtaining an accurate 
and appropriate result by having many different contributors performing a task but we'd like to start with a story which
we believe encapsulates the idea of crowdsourcing in its most positive and useful light.
\\
\\
In January 2009 Timothy Gowers (Fields Medal winner and avid blogger) used his blog to post a striking question \emph{Is massively 
collaborative mathematics possible?}. He posted a difficult and unsolved mathematical problem he was particularly interested in and invited 
people to contribute to its solution in the comments section. The project initally got off to a slow start but once the ice was broken the 
comments flooded in. 37 days , 27 contibuters and 800 comments later Timothy Gowers was able to announce that not only had they solved the 
original problem but they had also solved a harder generalisation of it, he called his experiment the \emph{Polymath Project} .
\\
\\
This is a nice example of how crowd sourcing can be used to combine the skills of many individuals and produce an answers to a complex problem,
however Crowd sourcing can be used to solve a wide variety of problems with a wide variety of motivations for instance:
\begin{itemize}
\item 
\emph{Image tagging} is an extemely arduous task for an individual or small group of people to perform and tagging a relatively large 
set of images could take weeks or even months, outsourcing this to the crowd could have the job done in a number of hours.
\item
\emph{Conducting surveys} can require asking a huge number of people many of whom are unwilling to take the survey in the first place
%FINISH THIS
\item
\emph{Searching for missing persons} could require a complex and costly ground operation in a foreign country which is hard
to co-ordinate. A crowd sourced search for aviator Steve Fosset involved salalite images being sent to the crowd with
users asking to flag suspcious objects which could look like a crash site.
\end{itemize}

This is all well and good in theory but it leaves us with a number of problems, an introduction to these problems is provided below but
they are discussed and addressed in much greater details at various stages throughout the report:
\begin{enumerate}
\item
\emph{How do we get these problems to the crowd?} It is unlikely many people with a problem to be solved would want to go to the trouble
of creating a crowd themselves and this be comparitive to the complexity of the problem itself therefore there is a need for a third party
crowd management system. 
\item
\emph{What problems can we ask the crowd?} Specialised crowds have been successful and certainly have their uses for instance
www.stackoverflow.com can be thought of as a crowd specialising in the solution of computing problems, however it is unlikely I would
be able to find a specialised crowd to indentify bird pictures or to search satellite images for crash sites, therefore I need access 
to a generalised crowd able to adapt to and solve a wide variety of problems. The crowd management party therefore needs to provide
the ability to ask a wide variety of questions and the ability to easily encorporate new questions in response to new technology. %EMBELISH?
\item
emph{How many people do we ask, who do we ask and how can we trust what they say?} These are all problems the crowd management
party faces, how many people do you have to ask to be sure of a cor 


\end{enumerate}
\subsection{A formal specification of our objectives}

\section{Design and Implementation}
% Detail your design why did you do it this way?
%Summarise Key implementation details (how did you do it what 
%tools did you use
 
\section{Evaluation}
%summarise testing procedures (+ relevant testing results)
%Evaluate your deliverables in terms of performance usability
%usefulness, (how successful was the project?)

\section{Conclusion and Future Extensions}
%Say what you've concluded from doing the work and how you'd build on it

\section{Project Management}
%Planning, group organisation, breakdown + task allocation etc

\section{Appendix}
%e.g user guide


\begin{thebibliography}{9}

\end{thebibliography}

\end{document}

